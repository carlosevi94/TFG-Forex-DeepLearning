


Este proyecto va a desarrollar una aplicaci'on final con un concepto que no estamos acostumbrados a'un a ver, el paradigma Data-Driven. 

Las aplicaciones que conocemos actualmente tienen actores y procesos, donde el actor suele ser el usuario final y los procesos son las acciones que el actor puede realizar. Toda la aplicaci'on se hace pensando siempre en el usuario final, siendo el elemento fundamental de 'esta. 

Para adaptar la filosof'ia Data-Driven y sus patrones de diseño necesitamos saber cómo usan los usuarios la aplicaci'on, y en base a este conocimiento se pueden desarrollar nuevas funcionalidades cruciales y adaptar funcionalidad existente para facilitar el uso al usuario.

Sin embargo, en la aplicaci'on que se va a desarrollar en este proyecto los actores no son los usuarios. Este escenario no es un escenario com'un, por lo que la aplicaci'on debe ir adaptada al verdadero actor del sistema,\textbf{ la red neuronal}. Dicha red es un algoritmo, por lo que no se puede comunicar con nosotros sobre algo que no sea lo que hemos desarrollado como salida. 

Este un hándicap para el diseño de toda la aplicaci'on, pero en especial para la capa de presentaci'on, ya que vamos a tener que desarrollar nosotros mismos la mejor forma de utilizar y visualizar los datos de la red sin ninguna referencia. 

Como soluci'on a este problema, hemos basado el diseño de la capa de presentaci'on en un panel de control m'as que en una aplicaci'on tradicional.  Este panel tiene interactividad con el usuario que consulta los datos que genera la red, pero no puede interactuar con la red en s'i. La red trabajar'a de forma aut'onoma, al igual que lo har'an los sistemas de almacenamiento y procesamiento de datos. 

















\clearpage