\chapter*{Glosario de términos}
\addcontentsline{toc}{chapter}{Glosario de t'erminos} 
 
\paragraph*{Accuracy} M'etrica para evaluar el rendimiento de un modelo de apredizaje autom'atico, se obtiene dividiendo el n'umero de aciertos entre el total.
\paragraph*{Activo financiero} T'itulo que representa para su poseedor derechos sobre bienes o rentas, y que es un pasivo para el agente que lo ha emitido.
\paragraph*{Algotrading} T'ecnica en la que se realizan operaciones sobre el mercado financiero automáticamente usando algoritmos.
\paragraph*{Análisis fundamental} Análisis que se realiza sobre un activo para determinar los factores que puede influir sobre su valor.
\paragraph*{Análisis técnico} Estudio de la acción del mercado.
\paragraph*{Aprendizaje no supervisado} T'ecnica de aprendizaje en la que se entrena un modelo sin datos de salida.
\paragraph*{Aprendizaje supervisado} T'ecnica de aprendizaje en la que se entrena un modelo con datos de salida.
\paragraph*{Batch} Número de ejemplos que se introducen a la red neuronal en cada época de entrenamiento.
\paragraph*{Big Data} Técnicas, herramientas y algoritmos que permiten trabajar sobre enormes vol'umenes de datos.
\paragraph*{Clasificación} Tipo de problema de aprendizaje autom'atico donde el objetivo es asignar una categor'ia a cada dato de entrada.
\paragraph*{Cluster} Conjunto de ordenadores que se comportan como si fuesen uno solo.
\paragraph*{Convolución} Operaci'on matem'atica donde se combinan dos funciones en una sola donde el resultado representa la magnitud que se superpone a la primera funci'on sobre una translaci'on invertida de la segunda funci'on.
\paragraph*{Data Science} Campo interdisciplinar que reúne las técnicas y conocimientos de los campos de la inteligenc'ia artificial, el big data, las matemáticas y la estad'istica para extraer conocimiento de conjuntos de datos.
\paragraph*{Deep learning} Subcategor'ia dentro del aprendizaje autom'atico que intenta modelar abstracciones de alto nivel en datos usando arquitecturas compuestas de transformaciones no lineales m'ultiples.
\paragraph*{Descenso del gradiente} Algoritmo iterativo de optimizaci'on para encontrar el m'inimo de una funci'on. Para encontrar este m'inimo, se calcula el gradiente en el punto actual y se avanza en la direcci'on negativa del gradiente calculado.
\paragraph*{'Epoca} Fase de entrenamiento de una red neuronal. Una 'epoca se completa cuando se han introducido todos los datos del entrenamiento tras dividirlos en batches.
\paragraph*{Gradiente} Pendiente de la tangente en un punto de una funci'on multivariable.
\paragraph*{IBEX 35} 'Indice burs'atil de la bolsa española formado por las 35 empresas con m'as liquidez. 
\paragraph*{Marca de tiempo} Secuencia que denota la hora y fecha en las que ocurre un determinado evento.
\paragraph*{Metamodelo} Modelo de un modelo. En machine learning se refiere a un modelo entrenado para generar otros modelos.
\paragraph*{MSE} El error cuadr'atico medio entre la estimaci'on y el valor real.
\paragraph*{Operación intradía} Operación de venta o compra de activos que se lleva a cabo si el activo alcanza cierto valor durante el día.
\paragraph*{Perceptr'on} Red neuronal que se compone de una 'unica neurona.
\paragraph*{Procesamiento de Lenguaje Natural} Rama de conocimiento de la inteligencia artificial que estudia las interacciones entre las m'aquinas y el lenguaje humano.
\paragraph*{Ratio de aprendizaje} Par'ametro asociado al entrenamiento de una red neuronal que indica la distancia de movimiento en la busqueda del m'inimo de la funci'on.
\paragraph*{Red neuronal} Modelo computacional basado en la interconexi'on de neuronas imitando la estructura de los cerebros org'anicos.
\paragraph*{Redes neuronales convolucionales} Tipo de red neuronal en la que se realiza operaciones de convoluci'on.
\paragraph*{Redes neuronales recurrentes} Tipo de red neuronal en las que existen ciclos entres las neuronas.
\paragraph*{Regresión} Tipo de problema de aprendizaje autom'atico donde la variable de respuesta.
\paragraph*{Sobreajuste} Efecto producido al sobreentrenar un modelo por el cual no es capaz de generalizar correctamente.
\paragraph*{Serie temporal} Secuencia de datos ordenados cronológicamente.
\paragraph*{Standalone} Modo de funcionamiento de sistemas basados en cluster para operar con un 'unico nodo.
\paragraph*{Software libre} Todo programa inform'atico cuyo c'odigo fuente pueda ser modificado y utilizado libremente.
\paragraph*{Web scraping} T'ecnica para la extracci'on de informaci'on de páginas web simulando la navegaci'on por ellas.



 


\clearpage